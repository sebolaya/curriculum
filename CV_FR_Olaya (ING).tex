\documentclass[11pt,a4paper]{moderncv}

\usepackage[FR]{myCV}

\usepackage[]{geometry}
  \geometry{top=.75cm, bottom=1.cm, left=1.5cm , right=1.5cm}


\addbibresource[label=biblio]{library.bib}


\begin{document}
\makecvtitle
\vspace{.5cm}
\section{\textbf{Mots clés}}
\cvitem{}{{\'Energies marines renouvelables}, {contr\^ole-commande}, {hydrodynamique}, {modélisation et optimisation numérique}, techniques de l'automatique avancée, {autonomie, rigueur}}

\section{\'Experiences Professionnelles}
\cventry{2017-(en poste)}{Ingénieur de Recherche}{D-ICE Engineering}{Nantes, France}{}{
    \begin{itemize}
        \item Prototypage (mod\'elisation + synth\`ese d'une loi de commande H$_{\infty}$ + $\mu$-analyse + estimateur d'\'etat) d'une solution permettant la suppression du ph\'enom\`ene de \og stick-slip\fg sur les trains de tiges utilisés dans le forage p\'etrolier
        \item \'Etude de faisabilit\'e (Capability plot) du système de propulsion d'une barge pour opération marine en shallow water
        \item Dévelopement du module (C++) de thrust allocation (formulation quadratique et non-linéaire) de navire dans un logiciel propriétaire de simulation d'opérations marines complexes
    \end{itemize}
}
\cventry{2016 (5 mois)}{Chercheur Postdoctoral}{\COER}{Maynooth, Irlande}{}{
    \begin{itemize}
        \item Application de la commande optimale (pseudo-spectrale) à un système houlomoteur
        \item Responsable de l'organisation des séminaires scientifiques du centre de recherche Marine Renewable Energy Ireland (MaREI)
    \end{itemize}
}
\cventry{2012-2015}{Ingénieur de Recherche}{\IRDL}{Brest, France}{}{
    \begin{itemize}
        \item Modélisation dynamique (mécanique + hydrodynamique) d'un système houlomoteur (projet FUI-12 \og EM BILBOQUET\fg{}) sous MATLAB/Simulink %(développement d'un code de calcul hydrodynamique propriétaire)
        \item Commande prédictive (MPC) en vue de la maximisation de l'énergie récupérée par le système
    \end{itemize}    
}
\cventry{2012-2015}{Vacataire}{\ENIB}{Brest, France}{}{%}
    \begin{itemize}
        \item Chargé de Td/Tp des modules de contr\^ole commande et d'\'electronique d'instrumentation
%        \item Chargé de Td/Tp, 5$^{ème}$ année - techniques de l'automatique avancée pour les systèmes linéaires
%        %systèmes linéaires, représentation d'état, placement de pôles, filtrage de Kalman, commande LQ/LQG.
%        \item Chargé de Td/Tp, 4$^{ème}$ année - électronique d'instrumentation, conversion d'énergie faible puissance
%        \item Concours ENI, Recrutement de première année
    \end{itemize}
}

\cventry{2009-2010}{Ingénieur R\&D}{IRIS-RFID}{Brest, France}{}{
    \begin{itemize}
        \item Hardware designer
%        \item Spécification des besoins matériels pour les différents lecteurs RFID UHF de l'entreprise
%        \item Développement de cartes électroniques (saisie de schéma, routage et prototypage)% et intégration mécanique
%%        \item Intégration mécanique
%        \item Mise en production (très faible série)
    \end{itemize}
}%

\cventry{2008-2009}{Contrat de professionnalisation}{IRIS-RFID}{Brest,~France}{}{
%    \begin{itemize}
%        \item Spécification des besoins et achat d'équipements d'instrumentation et de protoypage
%        \item Premiers développements sur les lecteurs RFID UHF
%    \end{itemize}
}%

\section{\textbf{Formation}}
\cventry{2012-2016}{\textbf{Doctorat}{~en Sciences de l'Ingénieur}}{\IRDL}{Brest, France}{}{Sujet: \textit{\og Contribution à la modélisation multi-physique et au contrôle optimal d'un générateur houlomoteur - Application à un système deux-corps\fg{}}}

\cventry{2010-2011}{Master 2 - Recherche}{Universit\'e de Bordeaux I}{Bordeaux, France}{position 2/15}{Spécialité: Automatique et Mécatronique pour les systèmes Automobiles, Aéronautiques et Spatiaux~(AM2AS)\\Sujet: \og\textit{Application de la commande sans modèle pour la régulation en vitesse variable d’une Machine à Courant Continu}\fg{}}

\cventry{2004-2009}{Diplôme d'ingénieur généraliste}{\ENIB}{Brest}{France}{
%    \begin{itemize}
%        \item 5$^{ème}$ année: Contrat de professionnalisation, \textit{IRIS-RID}, Brest, France.
%        \item 4$^{ème}$ année: Projet de fin d'étude (6 mois) - Développement d'un enregistreur de données pour un prototype d'éolienne suburbain de la société WINCAP ENERGY, Brest, France.
%    \end{itemize}
}

\section{\textbf{Compétences}}
\cvitem{Ingénierie}{\textit{Calcul/Modélisation}: MATLAB, Simulink, Scilab, NEMOH (ECN)}
%\cvitem{}{\textit{CAD}: Solidworks (mCAD) - Eagle, Altium Designer (eCAD)}
\cvitem{}{\textit{Langages/Outils}: C++, Python / Git}
%\cvitem{}{\textit{Outils de d\'evelopement}: Git}
%\cvitem{}{\textit{IDE microcontrôleur}: MPLAB (PIC), IAR, KEIL}
%\vspace{.25cm}
\cvitem{Bureautique}{LaTeX, MS Office}
%\vspace{.25cm}
\cvitem{Langue}{anglais (lu/écrit/parlé couramment)}

\section{\textbf{Centres d'intérêts}}
\cvitem{}{Hockey (en compétition), pêche, chasse sous-marine, bricolage, cuisine}
%\cvitem{Bricolage}{}

\end{document}
