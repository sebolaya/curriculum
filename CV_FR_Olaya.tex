\documentclass[18pt,a4paper]{moderncv}

\usepackage[FR]{myCV}

\usepackage[]{geometry}
%	\geometry{bottom=2cm}
  \geometry{top=1.5cm, bottom=2cm, left=1.5cm , right=1.5cm}


\addbibresource[label=biblio]{library.bib}


\begin{document}
\makecvtitle
\vspace{.5cm}
\section{\textbf{Mots clés}}
\cvitem{}{\textbf{\'Energies marines renouvelables}, système houlomoteur, \textbf{hydrodynamique}, modélisation et optimisation numérique, \textbf{contr\^ole-commande}}

%\newpage
\section{\'Experiences Professionnelles}
\cventry{2017-(en poste)}{Ingénieur de Recherche}{D-ICE Engineering}{Nantes, France}{}{
    \begin{itemize}
        \item Prototypage (mod\'elisation + synth\`ese d'une loi de commande H$_{\infty}$ + $\mu$-analyse + estimateur d'\'etat) d'une solution permettant la suppression du ph\'enom\`ene de \og stick-slip\fg sur les trains de tiges utilisés dans le forage p\'etrolier
        \item \'Etude de faisabilit\'e (Capability plot) du système de propulsion d'une barge pour opération marine en shallow water
        \item Dévelopement du module de thrust allocation (formulation quadratique et non-linéaire) de navire dans un logiciel propriétaire de simulation d'opérations marines complexes
    \end{itemize}
}
\cventry{juin-nov 2016}{Postdoc}{\COER, Maynooth University}{Maynooth, Ireland}{}{
    \begin{itemize}
        \item \textbf{Commande optimale pseudospectrale} pour systèmes houlomoteurs
    \end{itemize}
}
\cventry{2012-2016}{Doctorat~en Sciences de l'Ingénieur}{\LBMS}{Brest, France}{}{
    (voir la section \textit{Formation et Activités de Recherche} pour plus de détails)
    \begin{itemize}
        \item \textit{Savoir-faire} (sous \textbf{MATLAB/Simulink})
        \begin{itemize}
            %                \item[-] Énergie houlomotrice
            \item[-] Modélisation dynamique de \textbf{structure offshore} (théorie potentielle)
            \item[-] Résolution d'équation aux dérivées partielles (approche semi-analytique, code BEM: \textbf{NEMOH})
            \item[-] Commande optimale (\textbf{Model Predictive Control - MPC})
        \end{itemize}
        \item \textit{Savoir-être}
        \begin{itemize}
            \item[-] \textbf{Autonomie, rigueur}
        \end{itemize}
    \end{itemize}    
}
\cventry{2012-2015}{Vacataire}{\ENIB}{Brest, France}{}{}
\cvitemr{{\textit{108h}}}{\textit{Chargé de Td/Tp, 5$^{ème}$ année}, systèmes linéaires, représentation d'état, placement de pôles, filtrage de Kalman, commande LQ/LQG.}%
\cvitemr{\textit{103h}}{\textit{Chargé de Td/Tp, 4$^{ème}$ année}, électronique d'instrumentation, conversion d'énergie faible puissance.}%
\cvitemr{\textit{12h}}{\textit{Concours ENI}, Recrutement de première année.}
\cventry{2009-2010}{Ingénieur}{Hardware Designer}{\textit{IRIS-RFID}}{Brest, France}{
%    Au sein du département Recherche \& Développement, j'ai participé à la définition des différents lecteurs RFID UHF de la manière suivante:
    \begin{itemize}
        \item Spécification des besoins matériels
        \item Développement de cartes électroniques (saisie de schéma, routage et prototypage)
        \item Intégration mécanique
        \item Mise en production (très faible série)
    \end{itemize}
}%
\cventry{2008-2009}{Contrat de professionnalisation}{IRIS-RFID}{Brest,~France}{}{
%    Au sein du département Recherche et Développement de l'entreprise j'ai participé aux tâches suivantes:
%    \begin{itemize}
%        \item Spécification des besoins et achat d'équipements d'instrumentation et de protoypage
%        \item Premiers développements sur les lecteurs RFID UHF
%    \end{itemize}
}%

\section{\textbf{Compétences}}
\cvitem{Ingénierie}{\textit{Calcul/Modélisation}: MATLAB, Simulink, Scilab, NEMOH (ECN)}
%\cvitem{}{\textit{CAD}: Solidworks (mCAD) - Eagle, Altium Designer (eCAD)}
\cvitem{}{\textit{Langages/Outils}: C++, Python / Git}
%\cvitem{}{\textit{Outils de d\'evelopement}: Git}
%\cvitem{}{\textit{IDE microcontrôleur}: MPLAB (PIC), IAR, KEIL}
%\vspace{.25cm}
\cvitem{Bureautique}{LaTeX, MS Office}
%\vspace{.25cm}
\cvitem{Langue}{anglais (lu/écrit/parlé couramment)}

\newpage
\section{\textbf{Formation et Activités de Recherche}}
\cventry{2012-2016}{\textbf{Doctorat}{~en Sciences de l'Ingénieur}}{\LBMS}{Brest, France}{}{}
\cvitem{}{
	\underline{Titre}: \textit{Contribution à la Modélisation Multi-Physique et au Contrôle Optimal d'un Générateur Houlomoteur - Application à un Système \og Deux Corps\fg{}}.
}    
\cvitem{}{  
    \underline{Résumé}: Cette thèse s'est inscrite dans le cadre du 12$^\text{ème}$ appel à projet du \textit{Fonds Unique Interministériel} (FUI) lancé par l'État au premier semestre 2011.
    Le projet \og EM BILBOQUET\fg{} a été colabellisé par les pôles de compétitivité Mer Bretagne, Mer PACA et Tenerrdis.
    Il consiste en la réalisation d'un nouveau système de génération d'électricité issue du mouvement relatif entre deux corps flottants, mus par la houle.
    Dans cette thèse, nous nous sommes intéressés au contrôle optimal à appliquer sur la génératrice via les convertisseurs statiques, afin d'extraire le maximum d'énergie de la houle incidente.
    Dans un premier temps, nous avons établi les équations dynamiques régissant le comportement de la structure dans la houle en adoptant les hypothèses de la théorie potentielle.
    Pour ce faire, nous avons développé un code de calcul spécifique, basé sur une résolution du problème linéaire de tenue à la mer, par des méthodes dites semi-analytiques.
    Ce code de calcul permet de déterminer les coefficients hydrodynamiques nécessaires à l'écriture de l'équation dynamique dans le domaine fréquentiel, mais aussi dans le domaine temporel via une modification de la formulation de Cummins.
    Cette dernière nous permet ainsi, dans un second temps, de formuler le problème de maximisation de l'énergie récupérée comme un problème d'optimisation où la variable à optimiser est le couple résistant de la génératrice. Le problème est résolu en temps réel en adoptant une approche par algorithme dit à horizon fuyant.
}
\cvitem{}{ 
\underline{Mots clés}: Houlogénérateur, hydrodynamique, équation intégrale, calcul analytique, contrôle-commande, commande prédictive, otpimisation
}
\cvitem{}{  
    \underline{Jury}: (Thèse soutenue le 13 septembre 2016)
    \begin{itemize}
        \item[-] \textit{Président}: Guy \textsc{Clerc}, Professeur, {Université Lyon 1}
        \vspace{.2cm}
        \item[-] \textit{Rapporteur}: Alain \textsc{Clément}, Ingénieur de Recherche, {École Centrale de Nantes}
        \item[-] \textit{Rapporteur}: Tarek \textsc{Ahmed-Ali}, Professeur, {Université de Caen Basse-Normandie}
        \vspace{.2cm}
        \item[-] \textit{Examinateur}: Bernard \textsc{Molin}, Professeur Associé - HDR, {École Centrale de Marseille}
        \item[-] \textit{Examinateur}: John \textsc{Ringwood}, Professeur, {Maynooth University}, Ireland
        \vspace{.2cm}
        \item[-] \textit{Invité}: Marc \textsc{Le Boulluec}, Ingénieur, {IFREMER}
        \vspace{.2cm}
        \item[-] \textit{Co-encadrant}: Jean-Matthieu \textsc{Bourgeot}, Maître de Conférence, {ENIB}
        \item[-] \textit{Directeur de Thèse}: Mohamed E.-H. \textsc{Benbouzid}, Professeur, {\UBO}
    \end{itemize}
}

\cventry{2010-2011}{Master Recherche}{Universit\'e de Bordeaux I}{Bordeaux, France}{position 2/15}{
Automatique et Mécatronique pour les systèmes Automobiles, Aéronautiques et Spatiaux~(AM2AS)}
\cvitem{}{  
    \underline{Titre}: \textit{Application de la commande sans modèle pour la régulation en vitesse variable d'une Machine à Courant Continu}.
}
\cvitem{}{
    \underline{Résumé}: Dans le cadre du projet FUI \og EM BILBOQUET\fg{}, nous avons réalisé le dimensionnement d'une loi de commande, dite sans modèle ou $\mathbf{i}$-PID, pour une machine à courant continu (MCC) fonctionnant à vitesse variable de manière bidirectionnelle.
    La MCC devant par la suite servir d'actionneur d'entrainement dans un banc moteur constitué de la MCC et d'une machine synchrone ou asynchrone.
%    Ce dernier étant dédié à l'évaluation de nouvelles stratégies de commande dans un contexte d'énergies marines renouvelables.
    Du fait de cette contrainte de bidirectionnalité de l'entrainement, la machine se trouve fortement perturbée par des frottements, surtout à faible vitesse. Afin de s'en prévenir, nous avons proposé une nouvelle stratégie de commande, issue des travaux de M.~Fliess, C.~Join and M.~Mboup sur les estimateurs algébriques. Nous l'avons également comparé avec une commande plus classique (contrôleur PI + feedforward) afin d'évaluer ses performances.
}
\cvitem{}{
    \underline{Mots clés}: Machine à courant continu, commande sans-modèle, $\mathbf{i}$-PID, estimateurs algébriques
}
\cvitem{}{
    \underline{Encadrement}: Dr. Jean-Matthieu \textsc{Bourgeot} (encadrant), Prof. Mohamed E.-H. \textsc{Benbouzid} (co-encadrant)
}

\newpage
\cventry{2004-2009}{Diplôme d'Ingénieur}{\ENIB}{Brest}{France}{}
\cvitem{}{
    5$^{ème}$ année: Contrat de professionnalisation, \textit{IRIS-RID}, Brest, France.
}
\cvitem{}{
    4$^{ème}$ année: Projet de fin d'étude (6 mois) - Développement d'un enregistreur de données pour un prototype d'éolienne suburbain de la société WINCAP ENERGY, Brest, France.
}% END OF CVENTRY

\section{\textbf{Autres Activités de Recherche}}
\cvlistitem{Rapporteur pour les revues IEEE - \textit{Transactions on Sustainable Energy} et \textit{Transactions on Control Systems Technology}.}%
\cvlistitem{Membre élu des représentants des doctorants au conseil de laboratoire du LBMS (2012-2015)}%
\cvlistitem{Gestion des séminaires scientifiques pour le centre de recherche Marine Renewable Energy Ireland (MaREI)}


% Publications from a BibTeX file using the multibib package
\section{\textbf{Publications}}
\begin{refsection}
\nocite{Faedo2017, Olaya2015a,Olaya2015b,Olaya2014a,Olaya2014c,Olaya2014d,Olaya2013a}%
\cventry{}{\textsc{Articles dans une revue internationale avec comité de lecture}}{}{}{}{
\vspace{-.25cm}
\printbibliography[keyword=olaya,title={Articles de Revue},heading=none,type=article,env=nodis]%,
}% END OF CVENTRY
\cventry{}{\textsc{Conférences internationales avec comité de lecture}}{}{}{}{
\vspace{-.25cm}
\printbibliography[keyword=olaya,title={Articles de Conférence},heading=none,type=inproceedings,env=nodis]
}% END OF CVENTRY
\end{refsection}
\begin{refsection}
\nocite{Olaya2013b}%
\cventry{}{\textsc{Conférences nationales avec comité de lecture}}{}{}{}{
\vspace{-.25cm}
\printbibliography[keyword=olaya,title={Articles de Conférence},heading=none,type=inproceedings,env=nodis]
}% END OF CVENTRY
\end{refsection}

%\section{Personal Achievement and Interest}
%\cvitem{Sport}{roller in-line and ice hockey, diving and underwater fishing, judo}
%% \cvitem{Hobby}{electronic and micro}

\section{Références}
%%%\cventry{}{Encadrement de thèse}{}{}{}{
%%%    %\begin{minipage}{.5\linewidth}
%%%    \parbox[t]{.5\textwidth}{
%%%        Prof. Mohamed El-Hachemi BENBOUZID\\
%%%        Universit\'e de Bretagne Occidentale (UBO)\\
%%%        Rue de Kergoat - CS 93837\\
%%%        29238 Brest CEDEX 03, FRANCE\\
%%%        \emailsymbol~\emaillink{mohamed.benbouzid@univ-brest.fr}\\
%%%        \phonesymbol~+33~(0)2~98~01~80~07
%%%    }%\end{minipage}
%%%    %}
%%%    \hfill
%%%    \parbox[t]{.5\textwidth}{
%%%        %\begin{minipage}{.5\linewidth}
%%%        Dr. Jean-Matthieu BOURGEOT\\
%%%        \enib\\
%%%        945, Avenue du Technop\^ole\\
%%%        29280 Plouzan\'e, FRANCE\\
%%%        \emailsymbol~\emaillink{bourgeot@enib.fr}\\
%%%        \phonesymbol~+33~(0)2~98~05~66~28
%%%        \vfill
%%%    }%\end{minipage}
%%%}
%%%\vspace{.2cm}
%%%%\cventry{}{Responsable du module contrôle-commande}{}{}{}{
%%%%    %\begin{minipage}{.5\linewidth}
%%%%    \parbox[t]{.5\textwidth}{
%%%%        Prof. St\'ephane AZOU\\
%%%%        \enib\\
%%%%        945, Avenue du Technop\^ole\\
%%%%        29280 Plouzan\'e, FRANCE\\
%%%%        \emailsymbol~\emaillink{azou@enib.fr}\\
%%%%        \phonesymbol~+33~(0)2~98~05~66~44
%%%%    }%\end{minipage}
%%%%}

\cvitem{}{
\parbox[t]{.4\textwidth}{
Prof. Mohamed El-Hachemi BENBOUZID\\
Universit\'e de Bretagne Occidentale (UBO)\\
Rue de Kergoat - CS 93837\\
29238 BREST CEDEX 03, FRANCE\\
\phonesymbol~+33~(0)2~98~01~80~07\\
\emailsymbol~\emaillink{mohamed.benbouzid@univ-brest.fr}
}
\hspace{.5cm}%\separatorcolumnwidth}
\parbox[t]{.40\textwidth}{
Dr. Jean-Matthieu BOURGEOT\\
\ENIB\\
945, Avenue du Technop\^ole\\
29280 PLOUZAN\'E, FRANCE\\
\phonesymbol~+33~(0)2~98~05~66~28\\
\emailsymbol~\emaillink{bourgeot@enib.fr}
\vfill
}
}% END OF CVITEM


\end{document}
